
\section{Introduction}
\IEEEPARstart{T}{ime} series forecasting finds its applications across diverse domains, ranging from economic contexts for predicting stock prices to medical scenarios where it aids in forecasting health conditions or diseases over time. Additionally, it plays a crucial role in the realm of weather and climate forecasting, which is the focal point of this study. Specifically, this research delves into utilizing historical data from a weather station to predict future meteorological conditions, particularly the forecast of outdoor weather temperature. The main challenge lies in the amount of factors influencing temperature changes, such as wind, ocean currents, humidity, and other environmental variables. Addressing this problem is challenging not only due to the individual factors but also because of the intricate relationships among them.

In this particular scenario, the adoption of deep learning is deemed appropriate due to its capacity to effectively capture and model intricate relationships inherent in the dataset. The research methodology involves implementing a well-established deep learning model based on recurrent neural networks, with the aim of optimizing its performance by identifying the most suitable hyperparameters through the use of a Genetic Algorithms\cite{genetic-algorithms} for hyperparameter tuning. One notable limitation of deep learning models is their black box nature, where they operate on complex mathematical models that are interpretable only by computers, not humans. Recognizing this lack of interpretability, this study explores an alternative model that can deliver similar results to complex deep neural networks while providing interpretability. This alternative allows for a better understanding of how outputs are derived from inputs, offering transparency in the forecasting process.

